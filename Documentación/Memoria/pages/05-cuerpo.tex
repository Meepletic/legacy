\chapter{Introducción}

    \section{Motivación}

        Actualmente, los juegos de mesa han pasado a un segundo plano frente a otras formas de entretenimiento más modernas y dinámicas: los videojuegos, cada vez más realistas e inmersivos, y los servicios de series y películas, cuyo contenido se amplía tras cada estreno o exclusiva, son los principales alicientes de que los juegos de mesa poco a poco se vuelvan un tema menos frecuente del que hablar.

        Sin embargo, siguen existiendo personas cuya principal afición es juntarse alrededor de una mesa para jugar, comunicarse y mantener ese contacto e interacción que ofrece una partida en vivo.

        Precisamente por esa escasez de jugadores activos, sucede que: en bastantes ocasiones no se alcanza el número mínimo o adecuado de jugadores para un determinado juego de mesa; o por el contrario, hay jugadores más que suficientes pero tantos juegos donde elegir, que resulta tedioso tomar una decisión.

        Para resolver esos dos inconvenientes, se plantea crear una herramienta en formato de aplicación Android que facilite el descubrimiento (y posterior conexión) de personas con las que organizar partidas de juegos de mesa, para que puedan reunirse y jugar, y cuya principal función sea la de descubrir partidas cercanas mediante geolocalización.


    \section{Objetivos}

        Este trabajo de fin de grado pretende reducir o resolver los inconvenientes previamente planteados, creando una aplicación Android simple que permita a sus usuarios:

        \begin{itemize}

            \item Registrarse y establecer sus preferencias geográficas para descubrir partidas.
            \item Añadir y eliminar a otros usuarios a su lista de amigos.
            \item Añadir y eliminar juegos de mesa asociados a sus cuentas.
            \item Crear y eliminar partidas que usan los juegos de mesa asociados a las cuentas.
            \item Unirse a partidas organizadas por amigos u otros usuarios que hayan sido descubiertas por las preferencias geográficas.

        \end{itemize}

        \newpage


    \section{Estado del arte}

        Para la realización de este trabajo de fin de grado se ha empleado:

        \begin{itemize}

            \item Lua, un lenguaje de programación ligero, flexible y potente que es comúnmente utilizado en el desarrollo de videojuegos y aplicaciones Android.
            \item MySQL, un gestor de bases de datos común y de los más utilizados.
            \item IntelliJ IDEA, un entorno de programación creado por Jetbrains con amplias capacidades y uno de los mejores actualmente.
            \item DataGrip, un entorno de gestión de bases de datos, creado también por Jetbrains.

        \end{itemize}


    \section{Estructura de la memoria}

        Este documento está dividido en capítulos correspondientes a las fases de desarrollo que se establecieron para llevar a cabo el proyecto.

        % TODO

\cleardoublepage



\chapter{Investigación y estudio}

    Durante esta fase, se realizó una búsqueda de tecnologías y recursos que pudieran tener cabida en el proyecto de la aplicación Android.


    \section{Lua}

        Lua es un lenguaje de programación extensible, que ofrece buen soporte tanto para programación orientada a objetos, como para la programación funcional y la programación orientada a datos. Se busca que sea empleado como un lenguaje de script ligero pero potente, para que extienda cualquier programa que lo considere necesario: por tanto, al ser un lenguaje que extiende la función de otro, carece del concepto de programa principal, por lo que funciona embebido al programa anfitrión que lo utiliza.

        \subsection{Módulos empleados}

            Los módulos de Lua son como las librerías de cualquier lenguaje de programación.

            Este proyecto ha requerido de los módulos luasql-mysql para crear la conectividad entre Lua y la base de datos MySQL.


    \section{MySQL}


\cleardoublepage



\chapter{Diseño}

\cleardoublepage



\chapter{Implementación}

    \section{Aplicación Android}


    \section{Base de datos}


\cleardoublepage



\chapter{Pruebas, correcciones y mejoras}

\cleardoublepage



\chapter{Resultados}

    \section{Conclusiones}


    \section{Futuras líneas de trabajo}


\cleardoublepage
