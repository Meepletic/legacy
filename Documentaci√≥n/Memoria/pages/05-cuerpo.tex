\chapter{Introducción}

    \section{Motivación}
    
    Actualmente, los juegos de mesa han pasado a un segundo plano frente a otras formas de entretenimiento más modernas y dinámicas. Sin embargo, siguen existiendo personas cuya principal afición es juntarse alrededor de una mesa para jugar.
    
    Precisamente por esa escasez de jugadores activos, sucede que: en bastantes ocasiones no se alcanza el número mínimo o adecuado de jugadores para un determinado juego de mesa; o por el contrario, hay jugadores más que suficientes pero tantos juegos donde elegir, que resulta tedioso tomar una decisión.
    
    Para resolver esos dos inconvenientes, se plantea crear una herramienta que facilite el descubrimiento (y posterior conexión) de personas con las que organizar partidas de juegos de mesa para que puedan reunirse y jugar, cuya principal función sea la de descubrir partidas cercanas mediante geolocalización.
    
    
    \subsection{Objetivos}
    
    Este trabajo de fin de grado pretende reducir o resolver los inconvenientes previamente planteados, creando una aplicación Android simple que permita a sus usuarios:
    
    \begin{itemize}
        \item Registrarse y establecer sus preferencias geográficas para descubrir partidas.
        \item Añadir y eliminar a otros usuarios a su lista de amigos.
        \item Añadir y eliminar juegos de mesa asociados a sus cuentas.
        \item Crear y eliminar partidas que usan los juegos de mesa asociados a las cuentas.
        \item Unirse a partidas organizadas por amigos u otros usuarios que hayan sido descubiertas por las preferencias geográficas.
    \end{itemize}
    
    \newpage


    \section{Estudio del arte}
    
    Para la realización de este trabajo de fin de grado se ha empleado:
    
    \begin{itemize}
        \item Lua, un lenguaje de programación ligero, flexible y potente que es comúnmente utilizado en el desarrollo de videojuegos y aplicaciones Android.
        \item MySQL, un gestor de bases de datos común y de los más utilizados.
        \item IntelliJ IDEA, un entorno de programación creado por Jetbrains con amplias capacidades y uno de los mejores actualmente.
        \item DataGrip, un entorno de gestión de bases de datos, creado también por Jetbrains.
    \end{itemize}

\cleardoublepage



\chapter{Investigación y estudio}

    \section{Lenguaje de programación Lua}

        % \subsection{Módulos empleados}


    \section{Base de Datos MySQL}

\cleardoublepage



\chapter{Diseño}

\cleardoublepage



\chapter{Implementación}

\cleardoublepage



\chapter{Pruebas, correcciones y mejoras}

\cleardoublepage



\chapter{Resultados}

    \section{Conclusiones}


    \section{Futuras líneas de trabajo}


\cleardoublepage
